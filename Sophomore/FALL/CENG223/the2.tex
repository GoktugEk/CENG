\documentclass[11pt]{article}
\usepackage[utf8]{inputenc}
\usepackage{float}
\usepackage{amsmath}
\usepackage{amssymb}

\usepackage[hmargin=3cm,vmargin=6.0cm]{geometry}
%\topmargin=0cm
\topmargin=-2cm
\addtolength{\textheight}{6.5cm}
\addtolength{\textwidth}{2.0cm}
%\setlength{\leftmargin}{-5cm}
\setlength{\oddsidemargin}{0.0cm}
\setlength{\evensidemargin}{0.0cm}

% symbol commands for the curious
\newcommand{\setZp}{\mathbb{Z}^+}
\newcommand{\setR}{\mathbb{R}}
\newcommand{\calT}{\mathcal{T}}

\begin{document}

\section*{Student Information } 
%Write your full name and id number between the colon and newline
%Put one empty space character after colon and before newline
Full Name : Göktuğ Ekinci \\
Id Number : 2380343 \\

% Write your answers below the section tags
\section*{Answer 1}
\paragraph{a.}
\paragraph{i)}
According to the first property, empty set and A's itself can be in $\tau$. Also, expression has no contradictions with second and third properties. It \textbf{is a topology.}
\paragraph{ii)}
The union of the sets \{a\} and \{b\} is \{a,b\} which does not exist in $\tau$. This expression \textbf{is not a topology.}
\paragraph{iii)}
According to the first property, empty set and A's itself can be in $\tau$, so it does not contradict with first property. Also, all of the unions and intersections of the subsets of $\tau$ is in the $\tau$. This expression \textbf{is a topology.}
\paragraph{iv)}
The union of the subsets \{a,c\} and \{b\} of $\tau$ is \{a,b,c\} which is not in the $\tau$. This expression \textbf{is not a topology} according to the second property.
\paragraph{b.}
\paragraph{i)}
Assuming A is an infinite set, to achieve A-U = finite set, the U's are got to be infinite sets since the difference between infinite and finite sets is infinite. On the other hand, according to the third statement of topology, all intersections of set of U's have to be in the same set, but intersection of two infinite sets can be finite. If intersections of two infinite sets is finite, that shows these two sets do not meet when n goes to infinity, in other words, they do not have the same elements after any n. These kind of two sets cannot be in our sets of U's since they do not A-U = finite. Thus, \textbf{This expression is topology.}
\paragraph{ii)}
This proof is so similar to the first part. Assuming A uncountable set, to achieve A-U = countable set, the U's are got to be uncountable sets since the difference between uncountable and countable sets is uncountable. On the other hand, according to the third statement of topology, all intersections of set of U's have to be in the same set, but intersection of two uncountable sets can be countable. If intersections of two uncountable sets is countable, that shows these two sets do not meet when n goes to infinity, in other words, they do not have the same elements after any n. These kind of two sets cannot be in our sets of U's since they do not A-U = countable. Thus, \textbf{This expression is topology.}
\paragraph{iii)}
\textbf{This expression is not a topology on A,} here is the counter example.\\\\
$
A = N, U_1 = $Odd numbers, $U_2 = $Even numbers but not included 0\\
According to the second statement of topology, there must be a $U_3$ that is equal to:\\ 
$U_1 \cup U_2$\\
$N - U_3 = {0}$ which is not either $\emptyset$ or $\infty$
\textbf{This expression is not a topology.}

\section*{Answer 2}
\paragraph{a.}
$
f(a_1,b_1) = f(a_2,b_2) \\
a_1 + b_1 = a_2 + b_2\\
$
Since a's come from natural numbers and b's cannot be any integer value, these should be compared separately.\\
$
a_1 - a_2 = 0 \ and\  b_1 - b_2 = 0 \\
$
And this condition can be provided if and only if $a_1 = a_2\  and\  b_1 = b_2$ and this implies this \textbf{function is injective.}
\paragraph{b.}
This function is not surjective since it does not give any integers even integers are included in image set. For example, \\
if we accept any real number a + b as a, integer part, and b, real part between 0 and 1. If we try to produce an integer,  real part would not let us to produce any integer because it cannot take 0 or 1.
\paragraph{c.}
According to Schroeder–Bernstein theorem if there are two functions whose image and domain sets are A $\rightarrow$ B and B $\rightarrow$ A respectively. If these two functions are injective, their cardinality must be same. I proved that f is injective for its domain and image set. g is given injective for the same sets in reverse, thus \textbf{They have the same cardinality.}
\section*{Answer 3}
\paragraph{a.}
Mapping of \{0\} to $Z^+$ is infinitely countable since we do not change the number of elements in the set of $Z^+$. Mapping of \{0,1\} to $Z^+$ can be defined as $Z^+ \times Z^+$, and Cartesian product of two infinitely countable sets is also infinitely countable. \textbf{This set A is countable infinity.} 
\paragraph{b.}
As I explained that $Z^+ \times Z^+$ is infinitely countable. If we say $Z^+ \times Z^+$ is $(Z^+)^2$, we can say this expression equals to $(Z^+)^n$. Getting the product of two countable set is countable, so getting the product of countable sets (n-1) times does not change anything and it is also countable. \textbf{This set B is countable infinity.}
\paragraph{c.}
The set C includes the set of D, $D\subseteq C$. Cantor's second diagonal argument implies if a set is countable, all of its subsets must be countable too. There is a contradiction. \textbf{C is uncountable}
\paragraph{d.}
D is mapping to all values in \{0,1\}. In other words, it is the set of binary sequences, and according to Cantor's second diagonal argument, \textbf{D is uncountable.}

\paragraph{e.}
Let's say the place of the first 0 appears t. For t= 0 (situation where first number = 0), there are infinite possibilities for this situation. For t=1 (situation where second number is 0) and also there are infinite possibilities for this situation too, and this goes on for every possible natural number t. Thus, \textbf{Set E is uncountable according to the Cantor's second diagonal argument.}

\newpage
\section*{Answer 4}
\paragraph{a.}
By Stirling's Approximation $n! \approx \sqrt{2\pi n} * \frac{n^n}{e^n}$ \\
By theta's definition to arrive $n! = \theta(n^n)$, we must provide $c_1n^n< n!<c_2n^n $ where $c_1,c_2 \in R^+$ \\
Changing $n!$ with $\sqrt{2\pi n} * \frac{n^n}{e^n}$ \\
Here there is a contradiction since $c_1n^n <\sqrt{2\pi n} * \frac{n^n}{e^n} $ false for sufficiently big n.\\
So $\mathbf{n! \neq \theta(n^n)} $

\paragraph{b.}
If we open the expression $(n+a)^b$:\\\\
$(n + a)^b = n^b + c_1 * n^(b-1)*a^1 + c_1 * n^(b-2)*a^2 .... + a^b$ \\\\
As you can see for sufficiently big n's, this expression will asymptotically converge to $n^b$ since it is the highest ranked element of this expression. So, $(n + a)^b = \theta(n^b)$
\section*{Answer 5}
\paragraph{a.}
By definition we can write a = b \textbf{mod}c as b = cq + a. $(x^k - 1) = (x-1)(x^{k-1} + ...).$ As you can see,$(x-1)$ divides $(x^k -1)$. In this situation we change x with $2^y$. So, we left with $(2^y-1)$ divides $(2^{yk} -1)$, this leads us to $(2^x -1) \mathbf{mod} (2^y -1) = 2^a -1 = 2^{x\mathbf{mod}y} -1$
\paragraph{b.}
From the previous proof we can build up an equation like this:
$
gcd(2^{x} - 1, 2^b - 1) = gcd(2^y - 1, (2^{x+1} -1) \mathbf{mod}(2^y -1))\\
= gcd(2^y -1, 2^{(x+1)\mathbf{mod}y} -1)\\
= 2^{gcd(y,(x+1) \mathbf{mod}y) -1}\\$

And these leads us to :
$= 2^{gcd(x,y)} -1$


\end{document}