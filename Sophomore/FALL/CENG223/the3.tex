\documentclass[12pt]{article}
\usepackage[utf8]{inputenc}
\usepackage[dvips]{graphicx}
\usepackage{epsfig}
\usepackage{fancybox}
\usepackage{verbatim}
\usepackage{array}
\usepackage{latexsym}
\usepackage{alltt}
\usepackage{float}
\usepackage{amsmath}
\usepackage{hyperref}
\usepackage{listings}
\usepackage{color}
\usepackage[hmargin=3cm,vmargin=5.0cm]{geometry}
\topmargin=-1.8cm
\addtolength{\textheight}{6.5cm}
\addtolength{\textwidth}{2.0cm}
\setlength{\oddsidemargin}{0.0cm}
\setlength{\evensidemargin}{0.0cm}

\newcommand{\HRule}{\rule{\linewidth}{1mm}}
\newcommand{\kutu}[2]{\framebox[#1mm]{\rule[-2mm]{0mm}{#2mm}}}
\newcommand{\gap}{ \\[1mm] }

\newcommand{\Q}{\raisebox{1.7pt}{$\scriptstyle\bigcirc$}}

\lstset{
    %backgroundcolor=\color{lbcolor},
    tabsize=2,
    language=C++,
    basicstyle=\footnotesize,
    numberstyle=\footnotesize,
    aboveskip={0.0\baselineskip},
    belowskip={0.0\baselineskip},
    columns=fixed,
    showstringspaces=false,
    breaklines=true,
    prebreak=\raisebox{0ex}[0ex][0ex]{\ensuremath{\hookleftarrow}},
    %frame=single,
    showtabs=false,
    showspaces=false,
    showstringspaces=false,
    identifierstyle=\ttfamily,
    keywordstyle=\color[rgb]{0,0,1},
    commentstyle=\color[rgb]{0.133,0.545,0.133},
    stringstyle=\color[rgb]{0.627,0.126,0.941},
}


\begin{document}



\noindent
\HRule \\[3mm]
\small
\begin{tabular}[b]{lp{3.8cm}r}
{} Middle East Technical University &  &
{} Department of Computer Engineering \\
\end{tabular} \\
\begin{center}

                 \LARGE \textbf{CENG 223} \\[4mm]
                 \Large Discrete Computational Structures \\[4mm]
                \normalsize Fall '2020-2021 \\
                    \Large Homework 3 \\
                \normalsize Göktuğ Ekinci  \\
                \normalsize 2380343  \\
\end{center}
\HRule


\section*{Question 1}
According to the Fermat's Little Theorem we have 2 equalities: \\\\ 
1)$a^p = a \mathbf{mod}p$ \\\ 
2)$a^{p-1} = 1 \mathbf{mod}p$ if not p \vline\  a \\\\
We can write the equation like this: \\
${2^{11}}^2 + {4^{11}}^4 + {6^{11}}^6 + {8^{11}}^7 * 8^3 + {10^{11}}^{10}$ \\\\
According to the 1st equation we can convert this to this: \\

$2^2 + 4^4 + 6^6 + 10^{10} + 8^{10}$\\\\
According to the 2nd equation we can conver this to this: \\\\
$2^2 + 4^4 + 6^6 + 1 +1 \equiv 4 + 3 + 5 + 1 + 1 \equiv 14$\\

And $14 \equiv 3 \mathbf{mod}11$ answer is 3.

\section*{Question 2}
a = b $\times$ q + r where a,b,q,r are integers. gcd(a,b) = gcd(b,r)
$7n + 4 = (5n+3) * 1 + 2n + 1$\\\\
$5n+3 = (2n+1) * 2 + n+1 $\\\\
$2n + 1 = (n+1)*1 + n $\\\\
$ n+1 = n * 1 + 1$\\\\
$ n = 1 * n + 0$\\\\
Last remainder is 1, answer is 1


\section*{Question 3}

$m^2 = n^2 + kx$ then\\\\
$m^2 - n^2 = kx$ = $ (m-n) * (m+n) = kx$ this indicates that \\\\
$x\  \vline\ ((m-n) * (m+n))$ since x is a prime number and it has no divisors apart from itself and 1\\\\
Since it is in multiplication we can infer that\\\\
$ x\ \vline\ (m-n)$ or $x\ \vline\ (m-n)$ where x might divide both in the same time. 


\section*{Question 4}
For the base case 1: \\
$p(1) =\dfrac{1(3- 1)}{2} = 1$\\

Assuming p(n) = 1 + 4 + 7 ... (3n-2) = $\dfrac{n(3n-1)}{2}$ is true for all n $\geq$ 1. Then \\

p(n+1) = 1 + 4 + 7 ... 3(n+1) - 2  = $\dfrac{(n+1)(3(n+1) -1)}{2}$ \\

p(n+1) = 1 + 4 + 7 ... + 3n-2 + (3n +1)  = $\dfrac{(n+1)(3(n+1) -1)}{2}$ \\

p(n+1) = p(n) + 3n + 1 = $\dfrac{(n+1)(3(n+1) -1)}{2}$ replacing p(n) with $\dfrac{n(3n-1)}{2}$\\

$\dfrac{n(3n-1)}{2}$ + 3n + 1 =  $\dfrac{(n+1)(3(n+1) -1)}{2}$\\ 

$\dfrac{(n+1)(3(n+1) -1)}{2}$ = $\dfrac{(n+1)(3(n+1) -1)}{2}$\\

This proves the expression is valid for all n $\geq$ 1


\end{document}

